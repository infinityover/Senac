
\documentclass[14pt, a2paper, portrait]{tikzposter}

\usepackage[brazil]{babel}
\usepackage[utf8]{inputenc}
\usepackage[T1]{fontenc}
\usepackage{lipsum}
 
\newcommand{\bs}{\textbackslash}   
\newcommand{\cmd}[1]{{\bf \color{red}#1}}   
\tikzposterlatexaffectionproofon

\title{Sistema solar minimalista em {\LaTeX}}
\author{Paulo Belfi Dias da Silva\\ 
	    \texttt{paulobelfi@hotmail.com}}

\institute{Centro Universitário Senac}

 % Set colortheme
 % (default, anil, armin, edgar, emre, hanna, james, kai, lena, manuel,
 % martin, max, nicolas, pascal, peter, philipp, richard, roman, stefanie,
 % vinay)
\usecolortheme{hanna}

\definecolor{framecolor}{named}{black}

\settitlebodystyle{rectangular}
\setblocktitlestyle{rounded}
\setblockbodystyle{shaded}

\begin{document}

\titleblock[left fig=logo.png, embedded]

\block[l]{Introdução}{
	Desde o inicio da vida acadêmica do autor, o mesmo se fascinou pelo universo e seu funcionamento de uma forma geral.
   A ideia de desenvolver um projeto em que se simularia o movimento de translação dos planetas se deu por essa curiosidade sobre o universo, e como o tempo era escaço algumas variáveis tiveram que ser desconsideradas para que o projeto pudesse ser finalizado.
}

\begin{columns}
% 1a coluna
\column{0.48}

\block[c]{Materiais e Métodos}{
	Para desenvolver o projeto utilizei como base a imagem a baixo, pois é algo simples e auto explicativo.\linebreak
	\begin{center}
		\includegraphics[scale=0.3]{../exemplo.png}
	\end{center}
   Para o desenvolvimento da imagem base foram utilizados apenas funções básicas do pacote Tikz, draw e fill, e para gerar uma impressão de animação foi utilizado, o pacote beamer, gerando slides e fazendo a transição automática entre os slides.\linebreak
   
   Para gerar o código fonte do projeto foi criado um outro programa em linguagem C que dado o dia, gera a posição (x,y) de cada planeta e aplica em código  {\LaTeX}.\linebreak
   
   	Como o código fonte gerado, continha cerca de 966.00 linhas, e os editores não conseguiam abrir o mesmo, fez se necessário compila-lo diretamente pelo pdflatex, em um sistema Linux, pois o pdflatex do Windows se mostra inferior no quesito processamento. E mesmo desta forma, foram gastos quase 50 minutos para concluir a compilação do código fonte. Ao todo o são 60.140 slides, que demoram cerca de 17 minutos para completar a sua translação. Isso se deve ao fato de estar considerando o cada slide 1 dia terreste, e a translação de Netuno demorar o período de 164,79 anos.
   	
}
% 2a coluna
\column{0.52}

\block[c]{Resultado}{
   \begin{center}
   	\includegraphics[scale=0.6]{../final.png}
   \end{center}
}

\block{Considerações Finais}{
	Houveram diversas dificuldades no desenvolvimento deste projeto, sendo a sua maioria relacionadas ao tamanho do projeto e sua complexidade.\linebreak
	
	A maior parte da dificuldade se deu aos editores de código {\LaTeX} não suportarem o tamanho do arquivo que havia sido gerado.\linebreak
	
	A complexidade dentro do código {\LaTeX} é baixa, porem é uma alta quantidade de linhas a serem executadas, e alguns compiladores não conseguem executa-las. Fazendo com que fosse necessário encontrar alternativas para concluir o projeto.\linebreak
}

\end{columns}

\block[c,width=30cm]{Referências}{
\begingroup
   \renewcommand{\section}[2]{}
   \begin{thebibliography}{10}
	
	    \bibitem{Oetiker} OETIKER, Tobias et. al. {\sl Introdução ao {\LaTeXe}}, 2001.

	    \bibitem{Tantau} TANTAU, Till. {\sl The TikZ and PGF Packages}, http://www.texample.net/tikz/, 2014.
			
		\bibitem{Richter} RICHTER, Pascal et. al. {\sl The TikZposter class}, http://www.ctan.org/pkg/tikzposter/, 2014.

   \end{thebibliography}
\endgroup
}

\end{document}


